\documentclass[12pt,a4paper]{article}

\usepackage[margin=2.5cm]{geometry}
\usepackage{setspace}
\usepackage{parskip}
\usepackage{hyperref}

\onehalfspacing

\begin{document}
	
	\begin{center}
		\Large \textbf{ELEC-E8004 - Project Work} \\
		\Large \textbf{Team agreement + meeting summaries} \\
		\vspace{0.5cm}
		\normalsize
	\end{center}
	
	\section*{Part 1: Team Agreement}
	
	\subsection*{Team Members}
	\begin{itemize}
		\item Johan Riihimäki (Chairperson)
		\item Janita Sallanko
		\item Hoang Huy
		\item Santeri Vallin
	\end{itemize}
	
	\subsection*{Instructor}
	Phuoc Nguyen (present in weekly meetings and available via Telegram).
	
	\section*{1. Goals}
	
	\subsection*{a. Individual learning objectives}
	
	\textbf{Janita} \\
	Wants to learn robotics programming and gain hands-on experience with working mobile robots. Has a particular interest in mobile robotics and practical implementation.
	
	\textbf{Johan} \\
	??
	
	\textbf{Hoang} \\
	??
	
	\textbf{Santeri} \\
	??
	
	\subsection*{b. Personal grade goals}
	
	\begin{itemize}
		\item Janita: Grade 3
		\item Johan: ??
		\item Hoang: ??
		\item Santeri: ??
	\end{itemize}
	
	\section*{2. Getting Organized}
	
	\subsection*{a. Weekly meeting time}
	The team holds a fixed weekly meeting every Tuesday at 14:00, in person at the campus. The project instructor is also present during these meetings.
	
	\subsection*{b. Internal communication tools}
	Telegram is used for day-to-day communication within the team and for communication with the instructor.
	
	\subsection*{c. Storage of project materials}
	All course-related documents, code, notes, and other materials are stored in a shared GitHub repository, to which all team members have access.
	
	\section*{3. Chairperson}
	
	\subsection*{a. Chairperson model}
	The team has decided to have a fixed chairperson for the entire duration of the project to ensure continuity and efficient coordination.
	
	\subsection*{b. Chairperson}
	Johan serves as the chairperson for the whole project.
	
	\subsection*{c. Responsibilities of the chairperson}
	\begin{itemize}
		\item Preparing meeting agendas
		\item Facilitating discussions during meetings
		\item Supporting task organization and division of work
		\item Booking the robot required for project work
		\item Handling necessary project-related paperwork
	\end{itemize}
	The chairperson is not responsible for writing meeting memos.
	
	\subsection*{d. Responsibilities of the rest of the team}
	\begin{itemize}
		\item Active participation in meetings
		\item Following agreed agendas
		\item Supporting the chairperson
		\item Completing assigned tasks
		\item Communicating issues proactively
	\end{itemize}
	
	\section*{4. Work Practices}
	
	\subsection*{a. Fair task distribution}
	Tasks are distributed based on workload, skills, and learning goals. Responsibilities are tracked through discussions and via the shared GitHub repository.
	
	\subsection*{b. Decision-making}
	The team aims for consensus. If consensus cannot be reached, decisions are made by majority vote. In the case of a deadlock, the chairperson makes the final decision.
	
	\subsection*{c. Adhering to decisions and ground rules}
	Decisions are clarified during meetings, followed by clear task assignments and deadlines.
	
	\subsection*{d. Handling challenges}
	Challenges such as disagreements, communication issues, or deadline concerns are discussed openly within the team as early as possible. The instructor is involved if necessary.
	
	\subsection*{e. Instructor involvement}
	Issues are brought to the instructor’s attention when technical challenges block progress, workload concerns arise, conflicts cannot be resolved internally, or clarification is needed regarding course requirements.
	
	\section*{5. Team Atmosphere}
	
	\subsection*{a. Getting to know each other}
	Team members get to know each other through regular in-person meetings and collaborative work sessions.
	
	\subsection*{b. Maintaining a positive atmosphere}
	Meetings begin with brief updates on progress and challenges. Respectful communication and constructive feedback are emphasized.
	
	\subsection*{c. Encouraging participation}
	The chairperson actively invites input from all members. Telegram provides an additional channel for participation outside meetings.
	
	\newpage
	
	\section*{Part 2: Meeting Summaries}
	
	\subsection*{Meeting 1}
	
	\subsubsection*{1. Time \& Place}
	Date: Friday, 6 February \\
	Place: Campus (in person)
	
	\subsubsection*{2. Participants}
	\begin{itemize}
		\item Janita
		\item Johan
		\item Hoang
		\item Santeri
		\item Project instructor
	\end{itemize}
	
	\subsubsection*{3. Agenda}
	\begin{itemize}
		\item Individual discussions with the instructor
		\item Workload expectations
		\item Technical setup and environment configuration
		\item Questions related to project tools and robotics environment
	\end{itemize}
	
	\subsubsection*{4. Progress Update}
	The team had agreed on basic working practices, communication tools, and meeting schedules. Each team member discussed workload concerns and technical questions with the instructor. The instructor provided guidance on environment setup and clarified expectations for lab work.
	
	\subsubsection*{5. Challenges and Solutions}
	Uncertainty regarding technical setup was identified. The instructor addressed these issues individually and suggested documentation and support resources.
	
	\subsubsection*{6. Next Steps}
	\begin{itemize}
		\item Complete environment setup
		\item Begin familiarization with the robot and tools
		\item Start weekly Tuesday meetings at 14:00
	\end{itemize}
	
	\subsection*{Meeting 2: Weekly Instructor Meetings}
	
	\subsubsection*{1. Time \& Place}
	Tuesdays at 14:00 \\
	Campus (in person)
	
	\subsubsection*{2. Participants}
	Team members and project instructor.
	
	\subsubsection*{3. Agenda}
	\begin{itemize}
		\item Project progress
		\item Technical challenges
		\item Task division and planning
		\item Instructor feedback
	\end{itemize}
	
	\subsubsection*{4. Progress Update}
	Weekly meetings are used to review completed tasks, assign responsibilities, and receive instructor feedback.
	
	\subsubsection*{5. Challenges and Solutions}
	Challenges are discussed collectively. Solutions are identified through team discussion or instructor guidance.
	
	\subsubsection*{6. Next Steps}
	Continue project development, address challenges as they arise, and maintain weekly meetings until May or until instructed otherwise.
	
\end{document}
